%----------------------------------------------------------------%
%--------------------------INFORMATIONEN-------------------------%
%----------------------------------------------------------------%
%	Infos gibt es zu jedem Paket auf www.ctan.org
%	Werden bei den Paketen bestimmte Optionen gesetzt, so sind die Wichtigsten erklaert
%	solange sie nicht selbsterklärend sind

%----------------------------------------------------------------%
%--------------------------GRUNDEINSTELLUNGEN--------------------%
%----------------------------------------------------------------%
\documentclass[oneside, ngerman, footinclude=off, captions=tableheading, DIV=12, headerinclude=false, headings=optiontohead]{scrartcl}
\usepackage[bottom=2.5cm, inner=2.5cm, outer=2.5cm, top=2.5cm]{geometry}
%	'oneside'/'twoside': nicht zwischen linker und rechter Seite unterscheiden (alternativ twoside)
%	'twocolumn': wuerde 2 Spalten auf dem Blatt platzieren
%	'bibliography=totocnumbered': Normal nummeriertes Inhaltsverzeichnis (Kapitelnummer)
%	'listof=totocnumbered': Abbildungs- und Tabellenverzeichnis normal nummeriert (Kapitelnummer)
%	'ngerman' verwendet deutsch als Dokumentensprache (z.B. fuer Sirange)
%	'footinclude=off': Zaehlt Fusszeile zum Rand (vergroessert den Textbereich)
%	'captions=tableheading': Tabellenueberschriften explizit verwenden, erhoeht den Abstand zur Tabelle
%	'DIV=12': Kleinere Seitenraender (s. hierzu die KOMA-Documentation)

\usepackage[ngerman, english]{babel}							%	Einstellen der Sprache
\usepackage[T1]{fontenc}							%	Wie wird Text ausgegeben, d.h. im PDF
\usepackage[utf8]{inputenc}							%	Welche Zeichen 'versteht' LaTeX bei der Eingabe?
\usepackage{lmodern}								%	Laedt Schriften, die geglaettet sind
											
\usepackage{blindtext}								%	Beispieltext, zum Testen geeignet
%\usepackage{showframe}
%----------------------------------------------------------------%
%--------------------------ABSTÄNDE------------------------------%
%----------------------------------------------------------------%
\usepackage[onehalfspacing]{setspace}				%	Für Zeilenabstaende: 'singlespacing' (einfach), 'onehalfspacing' (1.5-fach), 'doublespacing' (2fach)
%\setlength{\parindent}{0cm}						%	Laengenangabe für die Einrueckung der ersten Zeile eines neuen Absatzes.
%\setlength{\parskip}{6pt plus 3pt minus 3pt}		%	Laengenangabe für den Abstand zwischen zwei Absaetzen.
%	Wenn diese beiden Befehle nicht kommentiert sind, wird ein Absatz nicht eingezogen sondern es gibt einen Abstand

%----------------------------------------------------------------%
%--------------------------MATHE---------------------------------%
%----------------------------------------------------------------%
\usepackage[]{mathtools}							%	Erweiterung von AMSMath, laedt automatisch AMSMath - für viele Mathe-Werkzeuge, 'fleqn' als Option ist für Mathe linksbuendig
\usepackage{amsfonts}								%	Für eine Vielzahl an mathematischen Symbolen

%----------------------------------------------------------------%
%--------------------------KOPF- UND FUSSZEILEN------------------%
%----------------------------------------------------------------%
\usepackage[automark,headsepline=.4pt]{scrlayer-scrpage}
\pagestyle{scrheadings}
\setkomafont{pageheadfoot}{\normalfont\bfseries}	%	Normale Schriftart und Fett für den Seitenkopf
\addtokomafont{pagenumber}{\normalfont\bfseries}	%	Normale Schriftart und Fett für die Seitenzahl

\clearscrheadfoot
\ohead{\thepage}									%	Rechter Seitenkopf mit Seitenzahl
\ihead{\headmark}									%	Linker Seitenkopf mit section
\ofoot[]{\empty}
%\ofoot{\thepage}									%	Leere Fußzeile, ungerade Seiten
%	Definert man oben in der documentclass 'twoside', so wird zwischen geraden und ungeraden Seiten unterschieden (NUR DANN!)

%----------------------------------------------------------------%
%--------------------------BILDER--------------------------------%
%----------------------------------------------------------------%
\usepackage{graphicx}									%	Um Bilder einbinden zu koennen 
\usepackage[dvipsnames,svgnames,table]{xcolor}			%	Farben verwenden, Versch. Farbdefinitionen, Farben in Tabellen (-Reihen, -Spalten)
\usepackage{pdfpages}									%	pdfs importieren
\definecolor{Seeblau100}{RGB}{0,169,224}				%	Uni-Farben, z.B. fuer Tabellen
\definecolor{Seeblau65}{RGB}{89,199,254}
\definecolor{Seeblau35}{RGB}{165,224,254}
\definecolor{Seeblau20}{RGB}{203,237,254}
\definecolor{Seegrau60}{RGB}{102,102,102}
\definecolor{Seegrau40}{RGB}{153,153,153}
\definecolor{Seegrau20}{RGB}{204,204,204}
\definecolor{Seegrau10}{RGB}{230,230,230}

%----------------------------------------------------------------%
%--------------------------POSITIONIERUNG------------------------%
%----------------------------------------------------------------%
\usepackage{float}

%----------------------------------------------------------------%
%--------------------------LISTEN--------------------------------%
%----------------------------------------------------------------%
\usepackage{enumitem}							%	Um Listen / Aufzaehlungen leichter zu modifizieren
%\setlist{noitemsep}							%	Verringert den Abstand in Aufzaehlungen

%----------------------------------------------------------------%
%--------TABELLEN-/BILDUNTERSCHRIFTEN und NUMMERIERUNG-----------%
%----------------------------------------------------------------%
\addtokomafont{captionlabel}{\bfseries}			%	Abbildung X.Y wir fett geschrieben
\setcapindent{2em}								%	2. Zeile teilweise haengend und eingezogen. Wenn ganz haengend gewuenscht, auskommentieren

\numberwithin{equation}{section}				%	Nummerierung der Gleichungen, Tabellen und Bilder nach der Kapitelnummer
\numberwithin{figure}{section}
\numberwithin{table}{section}

%----------------------------------------------------------------%
%--------------------------LITERATURVERZEICHNIS------------------%
%----------------------------------------------------------------%
\usepackage[german]{babelbib}					%	Bereitstellung des deutschen Layouts fuer die Bibliography

%----------------------------------------------------------------%
%--------------------------SIUNITX-------------------------------%
%----------------------------------------------------------------%
\usepackage[]{siunitx}
\sisetup{locale = DE}							%	Automatische Einstellung der Ausgabe für bestimmte Regionen (UK, US, DE, FR, ZA)

%----------------------------------------------------------------%
%--------------------------URLs / REFs---------------------------%
%----------------------------------------------------------------%
\usepackage[hidelinks]{hyperref}				%	Erweiterte Referenzierung ('hidelinks' verhindert Linien um Links)

%----------------------------------------------------------------%
%--------------------------EIGENE BEFEHLE------------------------%
%----------------------------------------------------------------%
\usepackage{mhchem}
%\usepackage{subfig}
\usepackage{subcaption}
\usepackage{listings}
%\lstset{numbers=none}
\definecolor{backcolour}{rgb}{0.95,0.95,0.92}
%\lstset{
%	numbers=none,
%	language=HTML,
%	breaklines=true,
%	frame=single,
%	tabsize=2,
	%backgroundcolor=\color{backcolour},  
%}

%\lstset{
%  language=HTML,
%  basicstyle=\ttfamily\small,
%  keywordstyle=\color{blue},
%  stringstyle=\color{red},
%  commentstyle=\color{green!50!black},
%  showstringspaces=false,
%  breaklines=true,
%  frame=single
%}

\definecolor{lightgray}{rgb}{.9,.9,.9}
\definecolor{darkgray}{rgb}{.4,.4,.4}
\definecolor{purple}{rgb}{0.65, 0.12, 0.82}

\lstdefinelanguage{JavaScript}{
  keywords={typeof, new, true, false, catch, function, return, null, catch, switch, var, if, in, while, do, else, case, break},
  keywordstyle=\color{blue}\bfseries,
  ndkeywords={class, export, boolean, throw, implements, import, this},
  ndkeywordstyle=\color{darkgray}\bfseries,
  identifierstyle=\color{black},
  sensitive=false,
  comment=[l]{//},
  morecomment=[s]{/*}{*/},
  commentstyle=\color{purple}\ttfamily,
  stringstyle=\color{red}\ttfamily,
  morestring=[b]',
  morestring=[b]"
}

\lstset{
   language=JavaScript,
   backgroundcolor=\color{lightgray},
   extendedchars=true,
   basicstyle=\footnotesize\ttfamily,
   showstringspaces=false,
   showspaces=false,
   numbers=left,
   numberstyle=\footnotesize,
   numbersep=9pt,
   tabsize=2,
   breaklines=true,
   showtabs=false,
   captionpos=b
}


\usepackage{makecell}